\begin{frame}
\frametitle{Sumário}
\tableofcontents
\end{frame}

\section{Problema do Transporte}

\begin{frame}
\frametitle{Definição do Problema do Transporte}

\begin{columns}[T]
    \begin{column}{0.6\textwidth}
        \begin{itemize}
            \item É uma classe especial dentro dos problemas de programação linear que trata da distribuição de mercadorias de várias origens para vários destinos;
            \item O objetivo é encontrar um plano de transporte que \textbf{minimize o custo total} de envio, respeitando as restrições de \textbf{oferta e demanda}.
        \end{itemize}
    \end{column}

    \begin{column}{0.5\textwidth}
        \centering
        \includegraphics[width=\linewidth]{images/Figura-1-Problema-do-transporte.png}
    \end{column}
\end{columns}
\end{frame}


\subsection{Exemplo Prático}

\begin{frame}
\frametitle{Exemplo Prático}
\begin{itemize}
    \item Uma empresa de calçados possui três fábricas e precisa distribuir sua produção para cinco lojas espalhadas pelo país. Cada fábrica possui uma capacidade máxima de produção semanal, e cada loja possui uma demanda específica a ser atendida.
    \item O custo de transporte de um par de sapatos entre cada fábrica e cada loja é conhecido e varia conforme a distância e o tipo de transporte utilizado.
    \item As capacidades de produção (oferta) das fábricas são: \textbf{200}, \textbf{300} e \textbf{250 pares}, respectivamente.
    \item As demandas das lojas são: \textbf{150}, \textbf{180}, \textbf{100}, \textbf{200} e \textbf{120 pares}, respectivamente.
\end{itemize}

\begin{center}
\textbf{Tabela de Custos de Transporte (R\$ por par)}
\end{center}

\begin{tabular}{c|ccccc|c}
\hline
 & L1 & L2 & L3 & L4 & L5 & Oferta \\ \hline
F1 & 8 & 6 & 10 & 9 & 7 & 200 \\
F2 & 9 & 12 & 13 & 7 & 5 & 300 \\
F3 & 14 & 9 & 16 & 5 & 8 & 250 \\ \hline
Demanda & 150 & 180 & 100 & 200 & 120 & \\ \hline
\end{tabular}
\end{frame}

\begin{frame}
\frametitle{Formulação do Problema}

\textbf{Restrições de Oferta (Capacidade de Produção):}
\[
\begin{aligned}
x_{11} + x_{12} + x_{13} + x_{14} + x_{15} &= 200 \\
x_{21} + x_{22} + x_{23} + x_{24} + x_{25} &= 300 \\
x_{31} + x_{32} + x_{33} + x_{34} + x_{35} &= 250
\end{aligned}
\]

\textbf{Restrições de Demanda (Necessidade das Lojas):}
\[
\begin{aligned}
x_{11} + x_{21} + x_{31} &= 150 \\
x_{12} + x_{22} + x_{32} &= 180 \\
x_{13} + x_{23} + x_{33} &= 100 \\
x_{14} + x_{24} + x_{34} &= 200 \\
x_{15} + x_{25} + x_{35} &= 120
\end{aligned}
\]

\textbf{Restrições de não negatividade:}
\[
x_{ij} \ge 0 \quad \forall i,j
\]
\end{frame}

\begin{frame}
\frametitle{Interpretação do Problema}
\begin{itemize}
    \item O objetivo é determinar quantos pares de sapatos cada fábrica deve enviar para cada loja de forma a \textbf{minimizar o custo total de transporte}.
    \item Todas as demandas devem ser atendidas, e nenhuma fábrica pode produzir mais do que sua capacidade máxima.
    \item Esse tipo de problema é conhecido como \textbf{Problema do Transporte Balanceado}, pois a soma das ofertas é igual à soma das demandas:
    \[
    200 + 300 + 250 = 150 + 180 + 100 + 200 + 120 = 750
    \]
    \item Pode ser resolvido utilizando métodos como o \textbf{Canto Noroeste}, \textbf{Vogel}, \textbf{Menor Custo}
\end{itemize}
\end{frame}

\subsection{Formulação Matemática}

\begin{frame}
\frametitle{Formulação Matemática}
\begin{itemize}
    \item \textbf{Variáveis de decisão:} 
    \[
    x_{ij} = \text{quantidade transportada da origem } i \text{ para o destino } j
    \]
    \item \textbf{Função objetivo:}
    \[
    \min Z = \sum_{i=1}^{m} \sum_{j=1}^{n} c_{ij}x_{ij}
    \]
    onde \(c_{ij}\) representa o custo de transporte por unidade.
    \item \textbf{Restrições:}
    \begin{align*}
    \sum_{j=1}^{n} x_{ij} &= a_i, && \forall i = 1, \dots, m \quad \text{(oferta)}\\
    \sum_{i=1}^{m} x_{ij} &= b_j, && \forall j = 1, \dots, n \quad \text{(demanda)}\\
    x_{ij} &\ge 0, && \forall i,j
    \end{align*}
\end{itemize}
\end{frame}

\subsection{Modelando o Exemplo}

\subsection{Métodos de Solução}

\begin{frame}
\frametitle{Métodos de Solução}
\begin{itemize}
    \item Método do Canto Noroeste;
    \item Método do Custo Mínimo;
    \item \textbf{Método de Vogel.}
\end{itemize}
\end{frame}

\subsection{Método de Vogel}

\begin{frame}
\frametitle{Método de Vogel}

\begin{itemize}
    \item O \textbf{Método de Vogel} é uma técnica heurística usada para encontrar uma \textbf{solução inicial viável} para o problema do transporte.
    \item Ele busca equilibrar \textbf{custo e penalidade}, tentando minimizar o custo total logo na alocação inicial.
    \item O princípio é penalizar as escolhas caras: a cada etapa, calcula-se o quanto se “perde” ao não escolher a opção mais barata de cada linha e coluna.
\end{itemize}
\end{frame}

\begin{frame}
\frametitle{Etapas do Método de Vogel}

\begin{enumerate}
    \item \textbf{Calcular as penalidades:}  
    Para cada linha e coluna, determine a diferença entre os dois menores custos dessa linha/coluna.  
    Essa diferença representa a \textbf{penalidade}.
    
    \item \textbf{Identificar a maior penalidade:}  
    Escolha a linha ou coluna com a penalidade mais alta (ou seja, onde o erro seria mais caro se não for escolhida a menor rota).
    
    \item \textbf{Selecionar a menor célula de custo} nessa linha ou coluna e alocar:
    \[
    x_{ij} = \min(\text{oferta}_i, \text{demanda}_j)
    \]
    
    \item \textbf{Atualizar as ofertas e demandas:}  
    Subtraia o valor alocado e risque a linha ou coluna esgotada.
    
    \item \textbf{Repetir o processo} até que todas as ofertas e demandas sejam atendidas.
\end{enumerate}
\end{frame}

\begin{frame}
\frametitle{Exemplo - Método de Vogel}
\begin{table}[]
\centering
\caption{Tabela de custos e demandas}
\begin{tabular}{c|ccc|c}
\hline
 & D1 & D2 & D3 & Oferta \\ \hline
O1 & 6 & 8 & 10 & 20 \\
O2 & 7 & 11 & 11 & 15 \\
O3 & 4 & 5 & 12 & 25 \\ \hline
Demanda & 10 & 10 & 40 &  \\ \hline
\end{tabular}
\end{table}
\end{frame}

\begin{frame}
\frametitle{Vogel — Iteração 1}
\begin{itemize}
  \item Penalidades (diferença entre dois menores custos):
    \begin{itemize}
      \item Linhas: $r_1=2,\; r_2=4,\; r_3=1$
      \item Colunas: $c_1=2,\; c_2=3,\; c_3=1$
    \end{itemize}
  \item Maior penalidade: \textbf{linha O2} ($r_2=4$). Menor custo nessa linha: $c_{2,1}=7$.
  \item Alocação: $x_{2,1}=\min(15,10)=10$.
\end{itemize}

\bigskip
\centering
\begin{tabular}{c|ccc|c}
 & D1 & D2 & D3 & Oferta \\ \hline
O1 & 6 & 8 & 10 & 20 \\
O2 & \textbf{7 (10)} & 11 & 11 & 5 \\
O3 & 4 & 5 & 12 & 25 \\ \hline
Demanda & 0 & 10 & 40 & 
\end{tabular}
\end{frame}

\begin{frame}
\frametitle{Vogel — Iteração 2}
\begin{itemize}
  \item Penalidades (após atualização):
    \begin{itemize}
      \item Linhas: $r_1=2,\; r_2=0,\; r_3=7$
      \item Colunas: $c_2=3,\; c_3=1$
    \end{itemize}
  \item Maior penalidade: \textbf{linha O3} ($r_3=7$). Menor custo nessa linha: $c_{3,2}=5$.
  \item Alocação: $x_{3,2}=\min(25,10)=10$.
\end{itemize}

\bigskip
\centering
\begin{tabular}{c|ccc|c}
 & D1 & D2 & D3 & Oferta \\ \hline
O1 & 6 & 8 & 10 & 20 \\
O2 & 7 (10) & 11 & 11 & 5 \\
O3 & 4 & \textbf{5 (10)} & 12 & 15 \\ \hline
Demanda & 0 & 0 & 40 & 
\end{tabular}
\end{frame}

\begin{frame}
\frametitle{Vogel — Iterações 3,4 e 5 (coluna D3)}
\begin{itemize}
  \item Agora resta apenas a coluna D3 (demanda 40).
  \item Passos realizados:
    \begin{enumerate}
      \item Alocamos $x_{3,3}=15$ (restante de O3).
      \item Alocamos $x_{2,3}=5$ (restante de O2).
      \item Alocamos $x_{1,3}=20$ (restante de O1).
    \end{enumerate}
\end{itemize}

\bigskip
\centering
\begin{tabular}{c|ccc|c}
 & D1 & D2 & D3 & Oferta \\ \hline
O1 & 6 & 8 & \textbf{10 (20)} & 0 \\
O2 & 7 (10) & 11 & \textbf{11 (5)} & 0 \\
O3 & 4 & 5 (10) & \textbf{12 (15)} & 0 \\ \hline
Demanda & 0 & 0 & 40 & 
\end{tabular}
\end{frame}

\begin{frame}
\frametitle{Solução inicial (Vogel) e Custo Total}
\[
X=
\begin{pmatrix}
0 & 0 & 20 \\
10 & 0 & 5 \\
0 & 10 & 15
\end{pmatrix}
\]

\bigskip
\begin{itemize}
  \item Cálculo do custo:
    \[
    Z = 7\cdot10 + 10\cdot20 + 5\cdot10 + 11\cdot5 + 12\cdot15 = 555
    \]
  \item Solução encontrada pelo Método de Vogel (alocação inicial): \(\boxed{Z=555}\).
\end{itemize}
\end{frame}

\begin{frame}
\frametitle{Exercício - Método de Vogel}
\begin{table}[]
\centering
\caption{Tabela de custos e demandas}
\begin{tabular}{c|cccc|c}
\hline
 & D1 & D2 & D3 & D4 & Oferta \\ \hline
O1 & 5 & 8 & 6 & 10 & 30 \\
O2 & 9 & 7 & 4 & 8 & 40 \\
O3 & 6 & 5 & 8 & 9 & 50 \\ \hline
Demanda & 20 & 30 & 25 & 45 & \\ \hline
\end{tabular}
\end{table}
\begin{block}{Enunciado}
Determine o plano de transporte ótimo utilizando o \textbf{Método de Aproximação de Vogel}, de modo a minimizar o custo total.
\end{block}
\end{frame}

\begin{frame}
\frametitle{Características do Método de Vogel}
\begin{itemize}
    \item Fornece soluções iniciais geralmente \textbf{mais próximas da ótima} do que o Método do Canto Noroeste ou o do Custo Mínimo;
    \item Requer cálculos adicionais, mas reduz o número de iterações posteriores no método de otimização (Simplex);
    \item É amplamente utilizado em problemas de transporte reais devido ao seu \textbf{bom equilíbrio entre precisão e esforço computacional}.
\end{itemize}
\end{frame}

\subsection{Método u-v (MODI)}

\begin{frame}
\frametitle{Método u-v (MODI)}

\begin{itemize}
    \item Também conhecido como \textbf{Método MODI} (\textit{Modified Distribution Method});
    \item É utilizado para \textbf{verificar a otimalidade} de uma solução viável inicial do problema do transporte;
    \item Permite identificar se é possível \textbf{reduzir o custo total} ajustando a alocação das variáveis não básicas;
    \item É uma forma simplificada de aplicar o \textbf{Simplex} ao problema do transporte.
\end{itemize}
\end{frame}

\begin{frame}
\frametitle{Ideia do Método u-v}

\begin{columns}[T]
\begin{column}{0.55\textwidth}
\begin{itemize}
    \item O método se baseia na atribuição de dois conjuntos de variáveis:
    \[
    u_i \text{ (para as origens) e } v_j \text{ (para os destinos)}
    \]
    \item Para cada célula básica \( (i,j) \), temos:
    \[
    c_{ij} = u_i + v_j
    \]
    \item As células não básicas são então avaliadas pelo \textbf{custo reduzido}:
    \[
    \Delta_{ij} = c_{ij} - (u_i + v_j)
    \]
\end{itemize}
\end{column}
\end{columns}
\end{frame}

\begin{frame}
\frametitle{Critério de Otimalidade}

\begin{block}{Condição de Ótimo}
Uma solução é \textbf{ótima} se e somente se:
\[
\Delta_{ij} \geqslant 0 \quad \forall (i,j) \text{ não básicos.}
\]
\end{block}

\begin{itemize}
    \item Se todos os \(\Delta_{ij}\) forem positivos ou nulos, o custo total atual é mínimo;
    \item Caso exista algum \(\Delta_{ij} < 0\), há oportunidade de melhoria no custo, a solução ainda não é ótima;
    \item Nesse caso, deve-se ajustar a alocação construindo um \textbf{ciclo fechado} para redistribuir as quantidades.
\end{itemize}
\end{frame}

\begin{frame}
\frametitle{Etapas do Método MODI}

\begin{enumerate}
    \item \textbf{Obter uma solução inicial viável} (ex: Método de Vogel);
    \item \textbf{Calcular os valores de \(u_i\) e \(v_j\):}
    \begin{itemize}
        \item Escolha arbitrariamente \(u_1 = 0\);
        \item Para cada célula básica, use \(c_{ij} = u_i + v_j\) para determinar os demais valores;
    \end{itemize}
    \item \textbf{Calcular os custos reduzidos:}
    \[
    \Delta_{ij} = c_{ij} - (u_i + v_j)
    \]
    \item \textbf{Verificar a otimalidade:}
    \begin{itemize}
        \item Se todos os \(\Delta_{ij} \geqslant 0\), pare, a solução é ótima;
        \item Caso contrário, vá para o próximo passo.
    \end{itemize}
\end{enumerate}
\end{frame}

\begin{frame}
\frametitle{Etapas do Método MODI (continuação)}

\begin{enumerate}
    \setcounter{enumi}{4}
    \item \textbf{Identificar a célula com o menor \(\Delta_{ij}\):}
    \begin{itemize}
        \item Essa será a célula \textbf{candidata a entrar na base};
    \end{itemize}
    \item \textbf{Construir o ciclo fechado:}
    \begin{itemize}
        \item Alterne entre células básicas, formando um caminho fechado;
        \item Sinalize os movimentos com \(+\) e \(-\);
    \end{itemize}
    \item \textbf{Atualizar as alocações:}
    \[
    x_{ij} = x_{ij} \pm \theta
    \]
    onde \(\theta\) é o menor valor das células marcadas com \(-\);
    \item \textbf{Repetir o processo} até que todos os \(\Delta_{ij} \geqslant  0\).
\end{enumerate}
\end{frame}

\begin{frame}
\frametitle{Exemplo - Método u-v (MODI)}

\begin{table}[]
\centering
\caption{Tabela de custos e solução inicial (via Vogel)}
\begin{tabular}{c|ccc|c}
\hline
 & D1 & D2 & D3 & Oferta \\ \hline
O1 & 6 & 8 & 10 & 20 \\
O2 & 7 & 11 & 11 & 15 \\
O3 & 4 & 5 & 12 & 25 \\ \hline
Demanda & 10 & 10 & 40 &  \\ \hline
\end{tabular}
\end{table}

\begin{block}{Objetivo}
A partir da solução inicial obtida pelo Método de Vogel, aplique o \textbf{Método MODI} para verificar se a solução é ótima e, se necessário, encontrar uma solução de custo menor.
\end{block}
\end{frame}

\begin{frame}
\frametitle{Exercício - Método u-v (MODI)}
\begin{table}[]
\centering
\caption{Tabela de custos e demandas}
\begin{tabular}{c|cccc|c}
\hline
 & D1 & D2 & D3 & D4 & Oferta \\ \hline
O1 & 10 (5) & 2 (10) & 20 & 11 & 15 \\
O2 & 12 & 7 (5) & 9 (15) & 20 (5) & 25 \\
O3 & 4 & 14 & 16 & 19 (10) & 10 \\ \hline
Demanda & 5 & 15 & 15 & 15 & \\ \hline
\end{tabular}
\end{table}
\begin{block}{Enunciado}
Determine o plano de transporte ótimo utilizando o \textbf{Método de Aproximação de Vogel}, de modo a minimizar o custo total.
\end{block}
\end{frame}

\begin{frame}
\frametitle{Vantagens do Método u-v (MODI)}

\begin{itemize}
    \item É o \textbf{método mais eficiente} para testar e melhorar soluções do problema do transporte;
    \item Evita o uso completo do \textbf{Simplex}, tornando o processo mais direto;
    \item Facilita a visualização da estrutura do problema;
\end{itemize}
\end{frame}

\begin{frame}
\frametitle{Resumo do Método u-v (MODI)}

\begin{block}{Passos Principais}
\begin{enumerate}
    \item Obtenha uma solução inicial viável (Vogel, Canto Noroeste, etc.);
    \item Calcule \(u_i\) e \(v_j\) com base nas células básicas;
    \item Determine \(\Delta_{ij}\) para as não básicas;
    \item Verifique se há \(\Delta_{ij} < 0\);
    \item Ajuste as alocações por ciclos fechados até atingir a otimalidade.
\end{enumerate}
\end{block}
\end{frame}

\subsection{Conclusão}
\begin{frame}
\frametitle{Conclusão}
\begin{itemize}
    \item O problema do transporte é uma aplicação clássica da programação linear com grande relevância prática;
    \item Seu estudo auxilia na compreensão de modelos de otimização com restrições de capacidade e custo;
    \item Métodos como o de Vogel e o MODI permitem soluções eficientes e economicamente viáveis.
\end{itemize}
\end{frame}


\subsection{Referências}
\begin{frame}
\frametitle{Referências}
\begin{itemize}
    \item OLIVEIRA, Jéssica Moia de. O problema de transporte. 2016. Projeto Supervisionado II (Graduação em Matemática, Estatística e Computação Científica) – Universidade Estadual de Campinas, Instituto de Matemática, Estatística e Computação Científica, Campinas, 2016. Orientador: Aurelio Ribeiro Leite de Oliveira.
    \item TAHA, Hamdy A. Operations research: an introduction. 10th ed. Global Edition. Harlow: Pearson Education, 2017.
    \item BYJU’S. Vogel’s Approximation Method. Disponível em: https://byjus.com/maths/vogels-approximation-method/
\end{itemize}
\end{frame}