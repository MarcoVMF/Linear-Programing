\documentclass{beamer}
\usepackage[utf8]{inputenc}
\usepackage[T1]{fontenc}
\usepackage[brazilian]{babel}
\usetheme{Madrid}

% Metadados do título
\title{Problema da Atribuição}
\author{Igor J. Rodrigues; Leonardo S. Coradeli; Lucas V. C. Ikeda; Marco V. M. Faria}
\institute{Universidade Estadual Paulista ``Júlio de Mesquita Filho''\\Faculdade de Ciência e Tecnologia}
\date{November 8, 2025}

\begin{document}

\logo{\includegraphics[height=0.5cm]{images/unesp.png}}

\frame{\titlepage}

% Slide 2: Definição
\begin{frame}{Definição}
  O problema da atribuição consiste em alocar um conjunto de agentes a um conjunto de tarefas, de modo que cada agente execute exatamente uma tarefa e vice-versa, minimizando o custo total dessa alocação.

  \vspace{0.5cm}
  \textbf{Exemplo:} Atribuir motoristas a rotas minimizando o tempo total de deslocamento.
\end{frame}

% Slide 3: Modelo Matemático
\begin{frame}{Modelo Matemático}
  \textbf{Variáveis de decisão:}
  $x_{ij} = \begin{cases}1 & \textrm{se o agente $i$ é atribuído à tarefa $j$} \\ 0 & \textrm{caso contrário} \end{cases}$
  \vspace{0.4cm}

  \textbf{Função objetivo:}
  \[
    \min \sum_{i=1}^n \sum_{j=1}^n c_{ij} x_{ij}
  \]

  \textbf{Restrições:}
  \[
    \sum_{j=1}^n x_{ij} = 1 ~ \forall i \qquad
    \sum_{i=1}^n x_{ij} = 1 ~ \forall j
  \]

  \textbf{Variáveis binárias:} $x_{ij} \in \{0, 1\}$
\end{frame}

% Slide 4: Representação em Tabela
\begin{frame}{Representação em tabela}
  Cada célula $c_{ij}$ representa o custo de atribuir o agente $i$ à tarefa $j$:

  \begin{center}
    \begin{tabular}{|c|c|c|c|}
      \hline
       & Tarefa 1 & Tarefa 2 & Tarefa 3 \\
      \hline
      Agente 1 & 15 & 20 & 17 \\
      Agente 2 & 13 & 18 & 21 \\
      Agente 3 & 19 & 14 & 16 \\
      \hline
    \end{tabular}
  \end{center}
  \vspace{0.3cm}
  Esta matriz é a base para a construção do modelo matemático do problema.
\end{frame}

% Slide 5: Obtenção de Solução
\begin{frame}{Obtenção de uma Solução}
  Para obter uma solução ótima, identificamos a combinação de atribuições agente-tarefa que leva ao menor custo total.

  \textbf{Exemplo prático:}
  \begin{itemize}
    \item Atribua os agentes às tarefas considerando cada combinação e calcule o custo total.
    \item Compare os custos e escolha a menor soma.
  \end{itemize}
\end{frame}

\begin{frame}{Somas das Atribuições Possíveis}
Considerando a matriz de custos:

\begin{center}
\begin{tabular}{|c|c|c|c|}
  \hline
   & Tarefa 1 & Tarefa 2 & Tarefa 3 \\
  \hline
  Agente 1 & 15 & 20 & 17 \\
  Agente 2 & 13 & 18 & 21 \\
  Agente 3 & 19 & 14 & 16 \\
  \hline
\end{tabular}
\end{center}

Possíveis somas (cada agente faz só uma tarefa):

\begin{itemize}
  \item Atribuição 1: 15 + 18 + 16 = 49
  \item Atribuição 2: 15 + 21 + 14 = 50
  \item Atribuição 3: 17 + 13 + 14 = 44
  \item Atribuição 4: 17 + 18 + 19 = 54
  \item Atribuição 5: 20 + 13 + 16 = 49
  \item Atribuição 6: 20 + 21 + 19 = 60
\end{itemize}
\textbf{Menor soma:} $44$, escolhendo Ag1-T3, Ag2-T1, Ag3-T2.
\end{frame}

% Slide: Método Manual pela Tabela
\begin{frame}{Método Manual pela Tabela}
Para problemas pequenos, a solução pode ser obtida diretamente pela matriz de custos:

\begin{itemize}
  \item Para cada linha ou coluna, escolha o menor valor ainda disponível.
  \item Atribua cada agente à tarefa correspondente, eliminando linha e coluna após cada escolha.
  \item Repita até que todos estejam alocados.
\end{itemize}

Este método é simples para matrizes pequenas, mas pouco eficiente em casos maiores.
\end{frame}

% Slide 6: Métodos de Resolução
\begin{frame}{Métodos de Resolução}
  Os principais métodos para resolver problemas de atribuição são:
  \begin{itemize}
    \item \textbf{Algoritmo Húngaro}: Método clássico e eficiente.
    \item Programação Linear: Pode ser resolvida pelo método simplex, mas o algoritmo húngaro é mais especializado.
    \item Métodos heurísticos: Variações para casos maiores ou com restrições especiais.
  \end{itemize}
\end{frame}

% Slide: Limitação do Simplex
\begin{frame}{Por que não usar só o Simplex?}
Apesar de o método simplex funcionar para problemas de atribuição, existem limitações importantes:

\begin{itemize}
  \item O número de variáveis e restrições cresce rapidamente, tornando inviável para problemas grandes.
  \item O Algoritmo Húngaro aproveita propriedades especiais da matriz de atribuição, resolvendo o problema em tempo polinomial, enquanto o simplex pode ser exponencial em certos casos.
  \item Na prática, o Húngaro é mais rápido, simples e especializado para atribuições.
\end{itemize}

\textbf{Conclusão:} Para problemas de atribuição, o Algoritmo Húngaro é claramente superior ao uso direto do simplex.
\end{frame}

% Slide: Modelo Simplex para Problema de Atribuição
\begin{frame}{Modelo Simplex para Problema de Atribuição}
Modelando com programação linear:

\begin{align*}
  & \min \sum_{i=1}^n \sum_{j=1}^n c_{ij}x_{ij} \\
  \text{sujeito a:} \qquad & \sum_{j=1}^n x_{ij} = 1 && \forall i = 1,\dots,n \\
                           & \sum_{i=1}^n x_{ij} = 1 && \forall j = 1,\dots,n \\
                           & x_{ij} \in \{0,1\} && \forall i, j
\end{align*}

O método simplex pode resolver este sistema, mas exige muito tempo computacional caso o número de agentes/tarefas seja grande.
\end{frame}

% Slide: Formulação Simplex - Variáveis e Função Objetivo
\begin{frame}{Resolução via Simplex --- Formulação}
Para a matriz apresentada, introduza variáveis $x_{ij}$ binárias:

\begin{align*}
  \min \quad & 15x_{11} + 20x_{12} + 17x_{13} \\
             & + 13x_{21} + 18x_{22} + 21x_{23} \\
             & + 19x_{31} + 14x_{32} + 16x_{33}
\end{align*}
Com as restrições:
\begin{align*}
  x_{11} + x_{12} + x_{13} = 1 &\quad \text{(Agente 1)} \\
  x_{21} + x_{22} + x_{23} = 1 &\quad \text{(Agente 2)} \\
  x_{31} + x_{32} + x_{33} = 1 &\quad \text{(Agente 3)} \\
  x_{11} + x_{21} + x_{31} = 1 &\quad \text{(Tarefa 1)} \\
  x_{12} + x_{22} + x_{32} = 1 &\quad \text{(Tarefa 2)} \\
  x_{13} + x_{23} + x_{33} = 1 &\quad \text{(Tarefa 3)} \\
  x_{ij} \in \{0,1\}
\end{align*}
\end{frame}

% Slide: Tableau Inicial e Interpretação
\begin{frame}{Resolução via Simplex --- Tableau Inicial}
Monte o tableau simplex para as variáveis:

\begin{table}
\centering
\begin{tabular}{c|ccccccccc|c}
\textbf{Base} & $x_{11}$ & $x_{12}$ & $x_{13}$ & $x_{21}$ & $x_{22}$ & $x_{23}$ & $x_{31}$ & $x_{32}$ & $x_{33}$ & $b$ \\
\hline
Agente 1 & 1 & 1 & 1 & 0 & 0 & 0 & 0 & 0 & 0 & 1 \\
Agente 2 & 0 & 0 & 0 & 1 & 1 & 1 & 0 & 0 & 0 & 1 \\
Agente 3 & 0 & 0 & 0 & 0 & 0 & 0 & 1 & 1 & 1 & 1 \\
Tarefa 1 & 1 & 0 & 0 & 1 & 0 & 0 & 1 & 0 & 0 & 1 \\
Tarefa 2 & 0 & 1 & 0 & 0 & 1 & 0 & 0 & 1 & 0 & 1 \\
Tarefa 3 & 0 & 0 & 1 & 0 & 0 & 1 & 0 & 0 & 1 & 1 \\
\hline
$z$ & 15 & 20 & 17 & 13 & 18 & 21 & 19 & 14 & 16 & 0 \\
\end{tabular}
\caption{Tableau inicial completo com linha do $z$ e restrições.}
\end{table}

\textbf{Observação}: Para problemas maiores o tableau cresce e pode ficar impraticável. Algoritmos especializados (Húngaro) resolvem com muito mais eficiência.
\end{frame}

% Slide 7: Algoritmo Húngaro
\begin{frame}{Algoritmo Húngaro}
  \textbf{Principais etapas:}
  \begin{itemize}
    \item Subtração do menor valor em cada linha e coluna.
    \item Marcação das linhas/colunas para identificar coberturas mínimas.
    \item Identificação das atribuições ótimas com custo zero.
  \end{itemize}

  \textbf{Propriedades:} Resolve o problema em tempo polinomial e garante a otimização.
\end{frame}

\begin{frame}{Algoritmo Húngaro: Passo 1}
Considere a matriz de custos:

\begin{center}
\begin{tabular}{|c|c|c|c|}
\hline
 & Tarefa 1 & Tarefa 2 & Tarefa 3 \\
\hline
Agente 1 & 15 & 20 & 17 \\
Agente 2 & 13 & 18 & 21 \\
Agente 3 & 19 & 14 & 16 \\
\hline
\end{tabular}
\end{center}

\textbf{1. Redução das linhas}

Subtraia o menor valor de cada linha:
\begin{itemize}
\item Agente 1 (menor = 15): $0, 5, 2$
\item Agente 2 (menor = 13): $0, 5, 8$
\item Agente 3 (menor = 14): $5, 0, 2$
\end{itemize}

Matriz reduzida pelas linhas:
\begin{center}
\begin{tabular}{|c|c|c|c|}
\hline
 & T1 & T2 & T3 \\
\hline
Ag1 & 0 & 5 & 2 \\
Ag2 & 0 & 5 & 8 \\
Ag3 & 5 & 0 & 2 \\
\hline
\end{tabular}
\end{center}
\end{frame}

\begin{frame}{Algoritmo Húngaro: Passo 2}
\textbf{2. Redução das colunas}

Subtraia o menor valor de cada coluna:
\begin{itemize}
\item Coluna 1 (menor = 0): não altera
\item Coluna 2 (menor = 0): não altera
\item Coluna 3 (menor = 2): subtraia 2
\end{itemize}

Matriz após redução das colunas:
\begin{center}
\begin{tabular}{|c|c|c|c|}
\hline
 & T1 & T2 & T3 \\
\hline
Ag1 & 0 & 5 & 0 \\
Ag2 & 0 & 5 & 6 \\
Ag3 & 5 & 0 & 0 \\
\hline
\end{tabular}
\end{center}
\end{frame}

% PASSO 3 - PARTE 1
\begin{frame}{Algoritmo Húngaro: Passo 3 (Parte 1)}
\textbf{Regra:} Procure linhas/colunas com apenas um zero. Marque esse zero e elimine sua linha e coluna.

\vspace{0.3cm}
\textbf{Matriz inicial após reduções:}

\begin{center}
\renewcommand{\arraystretch}{1.3}
\begin{tabular}{|c|c|c|c|}
\hline
 & T1 & T2 & T3 \\
\hline
Ag1 & \alert{0} & 5 & \alert{0} \\
Ag2 & \alert{0} & 5 & 6 \\
Ag3 & 5 & \alert{0} & \alert{0} \\
\hline
\end{tabular}
\end{center}

\textbf{Passo 1:} Linha Ag2 tem apenas \alert{1 zero} (coluna T1) $\to$ Marque \alert{Ag2-T1}

\textbf{Elimine linha Ag2 e coluna T1}
\end{frame}

% PASSO 3 - PARTE 2
\begin{frame}{Algoritmo Húngaro: Passo 3 (Parte 2)}
\textbf{Matriz após primeiro passo:}

\begin{center}
\renewcommand{\arraystretch}{1.3}
\begin{tabular}{|c|c|c|}
\hline
 & T2 & T3 \\
\hline
Ag1 & 5 & \alert{0} \\
Ag3 & \alert{0} & \alert{0} \\
\hline
\end{tabular}
\end{center}

\textbf{Passo 2:} Coluna T2 tem apenas \alert{1 zero} (linha Ag3) $\to$ Marque \alert{Ag3-T2}

\textbf{Elimine linha Ag3 e coluna T2:}

\begin{center}
\renewcommand{\arraystretch}{1.3}
\begin{tabular}{|c|c|}
\hline
 & T3 \\
\hline
Ag1 & \alert{0} \\
\hline
\end{tabular}
\end{center}

\textbf{Passo 3:} Só resta Ag1-T3 $\to$ Marque \alert{Ag1-T3}

\vspace{0.3cm}
\textbf{Atribuição completa:} \alert{Ag2-T1}, \alert{Ag3-T2}, \alert{Ag1-T3}

\textbf{Custo total:} $13 + 14 + 17 = 44$
\end{frame}

\begin{frame}{E se agentes $\neq$ tarefas?}
  O modelo clássico do problema da atribuição supõe que o número de agentes é igual ao número de tarefas.
  \vspace{0.4cm}
 
  Mas o que acontece se essa relação for diferente?
  \begin{itemize}
    \item Como lidar com cenários onde há mais agentes do que tarefas, ou vice-versa?
    \item O problema de atribuição ainda pode ser resolvido?
    \item Existem variantes do modelo clássico para essas situações?
  \end{itemize}
 
  \vspace{0.3cm}
  A seguir, veja como o problema é adaptado para esses casos.
\end{frame}


\begin{frame}{Casos com Número Diferente de Agentes e Tarefas}
  \textbf{Atribuição Desequilibrada:}
  \begin{itemize}
    \item Se o número de agentes $\neq$ número de tarefas, adiciona-se agentes ou tarefas fictícias na matriz de custos.
    \item Permite usar o mesmo modelo clássico, preenchendo as lacunas com custos nulos ou elevados.
  \end{itemize}
  
  \vspace{0.3cm}
  \textbf{Problema de Atribuição Generalizada (GAP):}
  \begin{itemize}
    \item Permite que um agente execute múltiplas tarefas, ou que tarefas tenham demandas específicas.
    \item Utiliza restrições adicionais de capacidade e exige técnicas diferentes de resolução, como branch and bound ou algoritmos heurísticos.
  \end{itemize}
\end{frame}

\begin{frame}{Atribuição Generalizada (GAP)}
  \textbf{O que é?}
  \begin{itemize}
    \item Variante do problema clássico onde cada agente pode executar múltiplas tarefas, respeitando restrições de capacidade.
    \item Cada tarefa deve ser atribuída exatamente a um agente, mas agentes podem acumular várias tarefas, até atingir seu limite.
  \end{itemize}
  
  \vspace{0.3cm}
  \textbf{Exemplo prático:}
  \begin{itemize}
    \item Alocação de entregas entre veículos, considerando a capacidade de carga de cada um.
    \item Distribuição de aulas entre professores, respeitando cargas horárias máximas.
  \end{itemize}

  \vspace{0.3cm}
  \textbf{Como resolver?}
  \begin{itemize}
    \item Utiliza Programação Inteira e métodos como branch and bound, heurísticas e relaxação lagrangeana.
    \item É considerado um problema difícil (NP-difícil) para instâncias grandes.
  \end{itemize}
\end{frame}



% Slide 8: Exercício Prático
\begin{frame}{Exercício}
  Considere a seguinte matriz de custos:
  \begin{center}
    \begin{tabular}{|c|c|c|c|}
      \hline
       & Tarefa 1 & Tarefa 2 & Tarefa 3 \\
      \hline
      Agente 1 & 8 & 7 & 9 \\
      Agente 2 & 6 & 5 & 8 \\
      Agente 3 & 7 & 9 & 6 \\
      \hline
    \end{tabular}
  \end{center}

  \textbf{Pergunta}: Usando o Algoritmo Húngaro, encontre a atribuição de menor custo e justifique cada passo.
\end{frame}

\begin{frame}{Resultado do Exercício}
  \textbf{Atribuição de menor custo:}
  
  \begin{itemize}
    \item Agente 1 $\rightarrow$ Tarefa 2 (custo 7)
    \item Agente 2 $\rightarrow$ Tarefa 1 (custo 6)
    \item Agente 3 $\rightarrow$ Tarefa 3 (custo 6)
  \end{itemize}
  
  \vspace{0.3cm}
  \textbf{Custo total mínimo:} $7 + 6 + 6 = 19$
\end{frame}


\nocite{*}

% Slide 10: Referências
\begin{frame}[allowframebreaks]{Referências}
  \bibliographystyle{plain}
  \bibliography{referencias}
\end{frame}

\end{document}
